\documentclass[12pt]{article}
\usepackage{esqu1}
\pagestyle{fancy}


\begin{document}
\section{Hermite Curves}
A \textbf{Hermite curve} is a curve defined by parametric cubic equations. Note that this does NOT mean that the curve itself is a portion of the graph of a cubic equation; it just means that the $x$ and the $y$ components of the curve are defined by cubic equations.

\[
\begin{aligned}
  x(t) &= a_xt^3 + b_xt^2 + c_xt + d_x \\
  y(t) &= a_yt^3 + b_yt^2 + c_yt + d_x
\end{aligned}
\]

Each Hermite curve is defined by four sets of points: 
\begin{itemize}
\item Two endpoints: $(x_0,y_0)$ and $(x_1,y_1)$
\item Two rate-of-change points that say how fast the curve is changing at the endpoints: $(dx_0,dy_0)$ and $(dx_1,dy_1)$.
\end{itemize}

The problem here is finding the coefficients $a,b,c,d$ from this set of points. 

We will derive the method here for the $x$ components only. For simplicity, we let the function be \[ f(t) = at^3 + bt^2 + ct + d \] \\ \\
When $t=0$, $f(0) = x_0 = d$ and $f'(0) = dx_0 = c$. \\
When $t=1$, $f(1) = x_1 = a+b+c+d$ and $f'(1) = dx_1 = 3a + 2b + c$. So:

\[ 
\begin{bmatrix}
x_0 \\ x_1 \\ dx_0 \\ dx_1
\end{bmatrix} =
\begin{bmatrix}
d \\ a+b+c+d \\ c \\ 3a + 2b + c 
\end{bmatrix} =
\begin{bmatrix}
0 & 0 & 0 & 1 \\ 1 & 1 & 1 & 1 \\ 0 & 0 & 1 & 0 \\ 3 & 2 & 1 & 0
\end{bmatrix} \times
\begin{bmatrix}
a \\ b \\ c \\ d
\end{bmatrix}
\]
\[
\begin{bmatrix}
a \\ b \\ c \\ d
\end{bmatrix}
=
\begin{bmatrix}
2 & -2 & 1 & 1 \\ -3 & 3 & -2 & -1 \\ 0 & 0 & 1 & 0 \\ 1 & 0 & 0 & 0
\end{bmatrix} \times
\begin{bmatrix}
x_0 \\ x_1 \\ dx_0 \\ dy_0
\end{bmatrix}
\]

where the latter matrix is the inverse of the former matrix. And that gives the coefficients of the parametric equation; so now you just have to graph the parametric, and you're done!

\section{Bezier Curves}
A \textbf{Bezier curve} is a curve with a certain degree. An $n$-degree Bezier curve is defined by $n+1$ points:
\begin{itemize}
\item Two endpoints, $P_0$ and $P_n$
\item $n-2$ ``tugging'' points, $P_1, P_2, \dots, P_{n-1}$
\end{itemize}

The parametric equation for the Bezier curve is (boldface letter represent matrix representations):
\[ {\bf B}_n(t) = t^n{\bf P}_n + \sum_{k = 0}^{n-1}(1-t)^{n-k}{\bf P}_k \]
The first few low-degree Bezier curves look like:
\[
\begin{aligned}
  \text{Linear}&: {\bf B}_1(t) = (1-t){\bf P}_0 + t{\bf P}_1 \\
  \text{Quadratic}&: {\bf B}_2(t) = (1-t)^2{\bf P}_0 + (1-t){\bf P}_1 + t^2{\bf P}_2 \\
  \text{Cubic}&: {\bf B}_3(t) = (1-t)^3{\bf P}_0 + (1-t)^2{\bf P}_1 + (1-t){\bf P}_2 + t^3{\bf P}_3
\end{aligned}
\]

\end{document}
